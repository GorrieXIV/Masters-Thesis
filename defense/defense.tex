\documentclass{beamer}
\usetheme{Singapore}

\usepackage{listings}
\usepackage{booktabs}
\usepackage{amssymb}                 %extended symbols
\usepackage{amsmath}                 %misc math formatting
\usepackage{algorithm}
\usepackage[noend]{algpseudocode}


\title[SIDH]{On the Efficiency of Isogeny-based Signatures Implementations}
\author{Robert W. V. Gorrie}
\institute[Comp Sci 2S03]{Department of Computing \& Software, McMaster University}
\date{Fall 2018}

\begin{document}

\begin{frame}
\titlepage
\end{frame}

\begin{frame}
\frametitle{Table of Contents}
\tableofcontents
\end{frame}

\AtBeginSection[]{
\begin{frame}<beamer>
\frametitle{Table of Contents}
\tableofcontents[currentsection]
\end{frame}
}

\section{Critical Background}

%high level signature description via fiat-shamir

\begin{frame}
\frametitle{Current Performance of SIDH}
\begin{center}
\begin{tabular}{@{}lllll@{}}
	Key Exchange Protocol & Speed & Overhead\\
	\midrule
	Code-based & ~0.5 ms & 360 KiB\\
	NTRU & 0.3-1.2 ms & 1 KiB\\
	Ring-LWE & 0.2-1.5 ms & 2-4 KiB\\
	LWE & 1.4 ms & v11 KiB\\
	\midrule
	SIDH & 15-400 ms & 0.5 KiB\\
\end{tabular}
\begingroup \fontsize{8pt}{8pt}\selectfont
https://s3.amazonaws.com/files.douglas.stebila.ca/files/research/presentations/20170918-QCrypt.pdf
\endgroup
\end{center}
\end{frame}

\begin{frame}
\frametitle{Isogeny Based Signatures}
\begin{itemize}
\item Yoo et. al provide an isogeny based signature scheme built off the Microsoft SIDH 1.0 Library.
\item The scheme is constructed using the Zero Knowledge Proof of Identity protocol provided in the original SIDH paper in tandem with Unruh's PQ secure Fiat-Shamir transform.
\item The scheme involves performing 248 (seperate) instances of SIDH key exchange with an arbitrary thirdparty
\item These instances are parallelizable but overall extremely computationally expensive
\end{itemize}
\end{frame}

\begin{frame}
\frametitle{Isogeny Signature Parameter Sizes}
\begin{tabular}{@{}lllll@{}}
	Scheme & Public-key size & Private-key size & Signature size\\
	\midrule
	Hash-based & 1,056 & 1,088 & 41,000\\
	Code-based & 192,192 & 1,400,289 & 370\\
	Lattice-based & 7,168 & 2,048 & 5,120\\
	Ring-LWE-based & 7,168 & 4,608 & 3,488\\
	Multivariate-based & 99,100 & 74,000 & 424\\
	\midrule
	Isogeny-base & 768 & 48 & 141,312\\
\end{tabular}\\
\begin{center}
\begingroup \fontsize{8pt}{8pt}\selectfont
https://eprint.iacr.org/2017/186.pdf
\endgroup
\end{center}
\end{frame}

\section{Inversion Batching}

%slide with each main point/part of the procedure 3 slides
\begin{frame}[fragile]
\frametitle{Batched Partial Inversions}
Inversions are traditionally one of the most expensive operations in modular arithmetic. We implement a procedure that further reduces the inversion count in this signature scheme.\\~\\

Our procedure utilizes three things:
\begin{itemize}
\item<1 -> The underlying parallel structure of the isogeny signature scheme
\item<2 -> Inversion batching
\item<3 -> Partial inversion of $\mathbb{F}_{p^{2}}$ elements
\end{itemize}
\end{frame}

\begin{frame}
\frametitle{Batching Inversions}
Consider first how we can restructure $n$ $\mathbb{F}_{p^{2}}$ inversions so that they can be done with 1 $\mathbb{F}_{p^{2}}$ inversion and roughly $3n$ $\mathbb{F}_{p^{2}}$ multiplications.
\end{frame}

%before this, background on inversions in sidh/sig
\begin{frame}[fragile]
\frametitle{Partial Inversion Procedure}
Consider now the following method for reducing one $\mathbb{F}_{p^{2}}$ inversion to 4 $\mathbb{F}_{p}$ multiplications and 1 $\mathbb{F}_{p}$ inversion.
\begin{center}
\begin{algorithm}[H]
\begin{algorithmic}[1]
	\Procedure{Partial $\mathbb{F}_{p^{2}}$ Inversion}{$a: (a_0,a_1)$} 
	\State $t_0 \gets a_{0}^{2}$
	\State $t_1 \gets a_{1}^{2}$
	\State $den \gets t_0 + t_1$
	\State $den \gets den^{-1}$
	\State $a_{0} \gets a_{0} * den$
	\State $a_{1} \gets a_{1} * den$
	\State \Return $a^{-1} \gets (a_{0},a_{1})$
	\EndProcedure
\end{algorithmic}
\end{algorithm}
\end{center}
%equation for inverting two elements, shamir?
%operation count reduction
\end{frame}

\begin{frame}[fragile]
\frametitle{Batched Inversion Procedure}
If we combine these two procedures we can reduce $n$ $\mathbb{F}_{p^{2}}$ inversions to:
\begin{itemize}
\item $1$ $\mathbb{F}_{p}$ inversion
\item $3(n-1)$ $\mathbb{F}_{p}$ multiplications
\item $2n$ $\mathbb{F}_{p}$ multiplications
\item $2n$ $\mathbb{F}_{p}$ squarings
\end{itemize}
Or, roughly 1 $\mathbb{F}_{p}$ inversion and 7n $\mathbb{F}_{p}$ multiplications
\end{frame}

\begin{frame}[fragile]
\frametitle{Performance Increase}
The following are measured in billions of clock cycles
\begin{center}
\begin{tabular}{@{}lllll@{}}
	\toprule
	Procedure & Without Batching & With Batching\\
	\midrule
	Signature Sign & 15.74 & 15.56\\
	Sign Parallel & 10.23 & 10.13\\
	Signature Verify & 11.18 & 10.8\\
	Verify Parallel & 7.27 & 7.11\\
	\bottomrule
\end{tabular}\\
\end{center}
\begin{itemize}
\item In the serial setting we see a 1.1\% and a 3.5\% performance increase for Signing and Verifying, respectively.
\item Comparatively, in the parallel setting we see a 0.9\% and a 2.3\% performance increase.
\end{itemize}
\end{frame}

\section{Signature Compression}

\begin{frame}
\frametitle{SIDH Key Compression}
A recent paper from Microsoft Research showed that SIDH public keys could be compressed to 330 bytes while retaining 128 bits of security in the quantum setting.\\~\\

These results, along with other changes, we're implemented in the second installment of the Microsoft SIDH library, which Yoo's signature scheme has yet to be subjected to.\\
\end{frame}

\begin{frame}
%goal
\frametitle{Signature Compression}
Because isogeny signatures are composed largely of SIDH public keys, Microsofts recently published techniques can offer further compression of these signatures.\\~\\

$$
\sigma \cong (\texttt{pk}_i, \texttt{sk}_i, h_i)
$$
\begin{center}for $1 \leq i \leq 248$\end{center}

Because public keys are distinctly seperate, we can use the same batched partial inversion procedure to cut down on time spent inverting $\mathbb{F}_{p^{2}}$ elements in compression and decompression.
\end{frame}

\section{Conclusive Remarks}

\begin{frame}
\frametitle{Performance Improvements Offered}
\end{frame}

\begin{frame}
\frametitle{Other Possible Performance Increases}
\begin{itemize}
\item Bos \& Friedberger have shown that using different parameters in the calculation of the underlying modulus can yield more efficient implementations for base field arithmetic.
\item Costello \& Hisil have recently published algorithms for efficiently computing arbitrary degree isogenies.
\item We hope to, through application of Costello and Hisil's work, explore the implications of Bos and Friedbergers findings on the SIDH library.\\
\end{itemize}
\end{frame}

\begin{frame}
\frametitle{Preparing SIDH 2.0 for Open Quantum Safe}
\begin{itemize}
\item Restructure and reorganize codebase with more ephasis on readability.
\item Further work to improve the efficiency of signing, verifying, compression, and decompression algorithms
\item We hope to work the protocol into an environment where it's practicality can more easily and assuredly be tested
\end{itemize}
\end{frame}

\begin{frame}


\begin{frame}
\begin{center}
SIDH 2.0 Fork with Signatures: https://github.com/GorrieXIV/PQCrypto-SIDH-gxivFork
\end{center}
\end{frame}

\end{document}
