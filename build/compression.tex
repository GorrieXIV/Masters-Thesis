\chapter{Compressing Signatures}
\label{sec:compress}

Our second contribution, also in the form of an addition to the \sidh signature extension, is a mechanism for compressing signatures. The following Chapter will cover the compression technique used. This Chapter, much like the last, will be split into three Sections: the first will cover the mathematics and high-level details of the compression technique used, the second will explain our implementation and integration of this technique into \sidh, and the third and final Section of this Chapter we will analyze in detail the results of this contribution.


\section{SIDH Key Compression Background}

Costello et al. showed in \cite{pkcomp} that  
We discussed rejection sampling A values from signature public keys until we found an A that was also the x-coord of a point. After some simple analysis, however, we found that it was extremely unlikely for A to be a point on the curve.

\subsection{Construction of Bases}

\subsection{Pohlig-Hellman}

\subsection{Decompression}

\section{Implementation Details}
\label{sec:compimplementation}

\subsection{Implementation Details}

\subsection{$\psi(S)$ Compression}

\noindent
\textit{Potential Tradeoffs}.

\section{Results}

Our technique can reduce the size of \sidh signature compression from \_\_\_ bits to \_\_\_ bits.

\subsection{Analysis}

\subsection{Potential Performance Improvements}

could use a non-constant implementation of double and add instead of montgomery's ladder? 

