\chapter{Conclusion}
\label{ch:conclusion}

In this chapter we provide our final set of metrics for the performance of the original isogeny-based signature scheme, our batched inversion implementation of the protocol, and our implementation feauturing $\psi(S)$ compression. We also offer measurements for how the compressed version of the protocol performs when combined with batched inversion.

Following the debriefing of our results, we offer one final section wherein we discuss the implications of our work in a general context. In this section we also discuss some possible future work to further progress the practicality of isogeny-based cryptography.

\section{Performance Results}

In this section we compile performance metrics for the original Yoo et al. signature scheme, our batched-inversion signature scheme, our compressed signature scheme, and also our combined compression with batched inversions implementation. For each of these implementations we show the average cycle time for \textbf{Sign} and \textbf{Verify} as well as the standard deviation. These measurements are outlined in \ref{fig:allmeasurements} (where ``C+B" denotes the combined compression with batching scheme). These averages are derived from 100 subsequent runs of each implementation.

\begin{table}
\begin{center}
\begin{tabular}{ | l | r | r | }
\hline
& Average Cycles & Standard Deviation \\
\hline
Original Sign & 4,950,023,141.654 & 300,643,097.882 \\
Original Verify & 3,466,703,991.096 & 263,674,018.528 \\
Batched Sign & 4,552,062,482.520 & 18,113,276.904 \\
Batched Verify & 3,173,340,239.461 & 68,672,478.339 \\
Compressed Sign & 10,224,610,996.644 & 465,349,640.468 \\
Compressed Verify & 4,472,444,449.556 & 182,317,386.709 \\
C+B Sign & 10,016,427,839.915 & 656,310,878.608 \\
C+B Verify & 4,326,294,567.596 & 175,349,338.690 \\
\hline
\end{tabular}
\end{center}
\caption{Average performance and standard deviation in clock cycles for all versions of the Yoo et al. signature scheme.}
\label{fig:allmeasurements}
\end{table}

We include graphical representations of our captured data, these can be found in \ref{fig:vanillacyclesgraph}, \ref{fig:batchedcyclesgraph}, \ref{fig:compressedcyclesgraph}, and \ref{fig:CBcyclesgraph} of Appendix \ref{app:perfdata}.

The reader might note that the the performance metrics of this protocol all yield a considerably high standard deviation. This can be attributed to a few factors. The first and perhaps most influential factor is the size of the randomly generated values such as the private key $m$. As these generated values fluctuate as does the time to compute field and point-wise arithmetic on them. This variance can only be attributed to non-constant time arithmetic, of course - which we have opted for in many cases due to the fact that we are mostly operating on public data.

On that same point, the reader will also note increased variance in the compressed implemetations. Part of this variance can be attributed to the fact that basis generation is a probabilistic process running in non-constant time - if favourable starting points are chosen, this process is completed significantly faster.

We also return again to the comparison charts first employed in Section \ref{subsec:perfcomparisons} to compare the temporal and spatial performance of these isogeny-based signatures to other post-quantum and classical alternatives. This time, we use the metrics resulting from our modified implementations as the point of comparison. These comparisons can be found in Table \ref{fig:endperfcomparisons} (comparing subroutine performances) and Table \ref{fig:endsizecomparisons} (compairing key and signature sizes). These metrics are all taken, yet again, at the 128-bit post quantum securty level (or 2048-bit and 256-bit classical security level, in the case of RSA and ECDSA).

\begin{table}[!h]
\begin{center}
\begin{tabular}{ l | r | r | r }
\hline
\mc{1}{}  & \mc{1}{Key Gen} & \mc{1}{Sign} & \mc{1}{Verify}\\
\hline
\rowcolor{Gray}
SIDH & 84,499,270 & 4,950,023,142 & 3,466,703,991 \\
\rowcolor{light-green}
SIDH Batched & 84,499,270 & 4,552,062,483 & 3,173,340,239 \\
\rowcolor{light-green}
SIDH Compressed & 84,499,270 & 10,224,610,997 & 4,472,444,450 \\
\rowcolor{light-green}
SIDH C+B & 84,499,270 & 10,016,427,840 & 4,326,294,568 \\
Sphincs & 17,535,886.94 & 653,013,784 & 27,732,049 \\
qTESLA & 1,059,388 & 460,592 & 66,491 \\
Picnic & 13,272 & 9,560,749 & 6,701,701 \\
\rowcolor{light-red}
RSA & 12,800,000 & 1,113,600 & 32400 \\
\rowcolor{light-red}
ECDSA & 1,470,000 & 128,928 & 140,869 \\
\hline
\end{tabular}
\end{center}
\caption{Performance in clock cycles for our improved isogeny-based signatures in comparison with other post-quantum and classical alternatives.}
\label{fig:endperfcomparisons}
\end{table}


\begin{table}[!h]
\begin{center}
\begin{tabular}{ l | r | r | r }
\hline
\mc{1}{}  & \mc{1}{Public Key} & \mc{1}{Private Key} & \mc{1}{Signature}\\
\hline
\rowcolor{Gray}
SIDH & 768 & 48 & 88,064 \\
\rowcolor{light-green}
SIDH Compressed & 768 & 48 & 69,632 \\
Sphincs & 32 & 64 & 8,080 - 16,976 \\
Rainbow & 152,097 - 192,241 & 100,209 - 114,308 & 64 - 104 \\
qTESLA & 4,128 & 2,112 & 3,104 \\
Picnic & 33 & 49 & 34,004 - 53,933 \\
\rowcolor{light-red}
RSA & 384 & 256 & 384 \\
\rowcolor{light-red}
ECDSA & 32 & 32 & 32 \\
\hline
\end{tabular}
\caption{Key and signature sizes for our compressed isogeny-based signatures in comparison with other post-quantum and classical alternatives.}
\label{fig:endsizecomparisons}
\end{center}
\end{table}

We report, as previously mentioned, roughly 8\% faster Yoo et al. signature signing and verifying when batching is implemented (and of course, this number can be increased if batching is implemented for the remaining inversion operations). We also note roughly 2\% faster signing for compressed signatures when batching is implemented, and 3\% faster verification.

Additionally, when we apply compression to Yoo et al. signatures we introduce another cross-thread inversion. This offers yet another avenue for implementing the partial batched inversion algorithm. We take advantage of this opportunity in our implementation, and our ``C+B" measurements reflect the results accordingly. Though compression offers another opportunity for batching inversions, the time spent on inversions (and thus the total time saved) becomes a much smaller percentage total execution time of the sign and verify algorithms (due to the intense computational overhead required to perform compression). This is why batching appears to offer less radical improvements to our compressed signature scheme.

\vspace{15px}

And so, we see from these comparisons that isogeny-based protocols can be improved upon through intelligent implementation. Our contributions have improved the size of Yoo et al. signatures by roughly 5\%, bringing them much closer to some implementations of the hash-based signature scheme Picnic.


\section{Discussion \& Concluding Remarks}

In this final section, we finish off the dissertation with some concluding remarks on the applicability of SIDH and isogeny-based cryptography, the importance of post-quantum cryptography, and the possible avenues for future work in this specific area.

\subsection{Future Work}
\label{sec:morebatch}

The next stage for this line of work is to finish applying inversion batching to the remaining cross-thread $\mathbb{F}_{p^2}$ inversions made throughout the signature scheme. There is 1 inversion call in both \textbf{Sign} and \textbf{Verify} that has yet to be processed for batching, and from which further (comparable) performance improvements can be made.

There are several other areas of the code-base where relatively simple changes could be made to gain marginal performance improvements. Take for example functions which previously ran on private information but have now been adopted to run on public information, such as the key-exchange functions used in the verification process. These functions are designed to employ constant time algorithms for performing arithmetic (such as the Montgomery ladder) but could now be modified to support non-constant time implementations. Changes here could save time at several points of the verification process.

Following further efforts to improve performance, the code-base should be heavily tested in terms of correctness and application security, and after continued scrutiny (and improvements to code design,) a pull request can be made to the Microsoft SIDH repository \cite{sidhcode} to merge both the Yoo et al. signature scheme and our improved implementations into their code-base.

In addition to all of this, there is one other obvious setting in which the inversion batching technique of Chapter \ref{sec:batching} could be leveraged for performance improvements. Consider some web domain servicing many end-users in parallel over HTTPS (or any secure communication protocol in which isogeny-based cryptography could be deployed). Said web server could, with high enough traffic, batch together inversion calculations from separate SIDH or Yoo et al. signature scheme implementations with many different users so as to decrease the amount of time spent on field element inversions.

Parallel to this line of work, there are of course the continued efforts of mathematically-inclined researchers to produce alternative designs for isogeny-based signature schemes, and alternative isogeny-based schemes that offer solutions to other information security goals. With advancements in algorithm and cryptosystem design happening in parallel with research on intelligent design, isogeny-based schemes (and post-quantum cryptography at large) will continue to approach practical and deployable systems.


\vspace{50px}

To conclude, implementations of cryptographic primitives do have a lot to gain from intelligent design and implementation when it comes to performance metrics. Classical cryptographic algorithms have been targetted by research of this sort for many decades - post-quantum systems on the other hand have younger and perhaps less optimized implementations. We believe that as the underlying foundations of post-quantum protocols gain traction and wide-spread confidence, more developers will begin to experiment with these protocols and the number of alternative implementation mechanisms and techniques will flourish, offering variety in terms of time-space tradeoffs and efficient, system-specific implementations.

As mathematical and developmental research both continue to provide more efficient and secure implementations of post-quantum protocols, we can continue to approach a cryptographically secure world in the face of a rapidly developing cryptanalytic threats.
