\chapter{Discussion \& Conclusion}

If one assumes security of the original SIDH key exchange protocol, then the Yoo et al. signature scheme is provably secure and requires no additional underlying assumptions. Given that \sidh there are non-ephemeral, and so could pose a promising candidate in the context of TLS certificate signing. Certificate authority signatures, not requiring constant transfer over the wire (as they are offen packaged with Internet browsers or sperating systems) are less desperate for small signature sizes. 

\section{Results \& Comparisons}

\subsection{Combining Batching \& Compression}

\begin{center}
\begin{tabular}{l | b | b | b }
\hline
\mc{1}{}  & \mc{1}{Key Gen} & \mc{1}{Sign} & \mc{1}{Verify}\\
\hline
\rowcolor{Gray}
SIDH & a & b & c \\
Sphincs & a & b & c \\
Rainbow & a & b & c \\
qTESLA & a & b & c \\
Picnic & a & b & c \\
\rowcolor{light-red}
RSA & a & b & c \\
\rowcolor{light-red}
ECDSA & a & b & c \\
\hline
\end{tabular}
\end{center}

\begin{center}
\begin{tabular}{l | b | b | b }
\hline
\mc{1}{}  & \mc{1}{Public Key} & \mc{1}{Private Key} & \mc{1}{Signature}\\
\hline
\rowcolor{Gray}
SIDH & 768 & b & 141,312 \\
Sphincs & 32 & 64 & 8,080 - 16,976 \\
Rainbow & 152,097 - 192,241 & 100,209 - 114,308 & 64 - 104 \\
qTESLA & 4,128 & 2,112 & 3,104 \\
Picnic & 33 & 49 & 34,004 - 53,933 \\
\rowcolor{light-red}
RSA & 384 & b & 384 \\
\rowcolor{light-red}
ECDSA & 32 & b & 32 \\
\hline
\end{tabular}
\end{center}

\section{Additional Opportunities for Batching}
\label{sec:morebatch}

\section{Future Work}
