\vspace{2in}
\begin{abstract}

\commentblock{basic introduction to the field, comprehensible to any technical person}
Progress in the field of quantum computing has shown that, should construction of a sufficiently powerful quantum computer become feasible, much of the cryptography used on the Internet today will be rendered insecure. In lieu of this, several approaches to ``quantum-safe" cryptography have been proposed, each one becoming a serious field of study. \commentblock{more detailed background, comprehensible to any person in a related discipline} The youngest of these approaches, isogeny based cryptography, is oriented around problems in algebraic geometry involving a particular variety of elliptic curves. Supersingular isogeny Diffie-Hellman (SIDH) is this subfield's main contender for quantum-safe key-exchange. Yoo et al.\ have provided an isogeny-based signature scheme built on top of SIDH. \commentblock{state general problem being addressed} Currently, cryptographic algorithms in this class are hindered by poor performance metrics and, in the case of the Yoo et al.\ signature scheme, large communication overhead.

\commentblock{summarise the main result (use the words "here we show" or something equivalent)}
In this dissertation we explore two different modifications to the implementation of this signature scheme; one with the intent of improving temporal performance, and another with the intent of reducing signature sizes. We show that our first modification, a mechanism for batching together expensive operations, can offer roughly $1.1$\% faster signature signing, and roughly $3.5$\% faster signature verification. Our second modification, an adaptation of the SIDH public key compression technique outlined in \cite{pkcomp}, can reduce Yoo et al.\ signature sizes from roughly $928\lambda$ bytes to $640\lambda$ bytes at the $128$-bit security level on a $64$-bit operating system. We also explore the combination of these techniques, and the potential of employing these techniques in different application settings. \commentblock{explain what the main result reveals in direct comparison to what was thought to be the case previously, or how the main result adds to previous knowledge} Our experiments reveal that isogeny based cryptosystems still have much potential for improved performance metrics. While some practitioners may believe isogeny-based cryptosystems impractical, we show that these systems still have room for improvement, and with continued research can be made more efficient - and eventually practical. \commentblock{provide a broader perspective, readily comprehensible to any technical person} Achieving more efficient implementations for quantum-safe algorithms will allow us to make them more accessible. With faster and lower-overhead implementations these primitives can be run on low bandwidth, low spec devices; ensuring that more and more machines can be made resistant to quantum cryptanalysis.

\end{abstract} 
