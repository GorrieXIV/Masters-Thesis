\vspace{2in}
\begin{abstract}

\commentblock{basic introduction to the field, comprehensible to any technical person}
Progress in the field of quantum computing has shown that, should construction of a sufficiently powerful quantum computer become feasible, much of the cryptography used on the Internet today will be rendered insecure. In lieu of this, several approaches to ``quantum-safe" cryptography have been proposed and become a serious field of study. \commentblock{more detailed background, comprehensible to any person in a related discipline} The youngest of these approaches, isogeny based cryptography, is oriented around problems in algebraic geometry concerned with a particular variety of elliptic curves. Supersingular isogeny Diffie-Hellman (SIDH) is this subfields main contender for quantum-safe key-exchange. Yoo et al. have also provided an isogeny-based signature scheme built on top of SIDH. \commentblock{state general problem being addressed} This class of cryptography has been hindered by poor performance metrics and, in the case of the Yoo et al. signature scheme, large communication overhead.

\commentblock{summarise the main result (use the words "here we show" or something equivalent)}
In this dissertation we explore two different modifications to the implementation of this signature scheme; one with the intent of improving temporal performance, and another with the intent of reducing signature sizes. We show that our first modification, a mechanism for batching together expensive operations across threads, can offer roughly  $x$ times faster signature signing, and $y$ times faster signature verification. Our second modification, an adaptation of the recently published public key compression algorithm for SIDH public keys or isogeny based signatures, can reduce signature sizes from ~$x$ bytes to ~$y$ bytes at the $z$ bit security level. We also explore the combination of these technique, and the potential of employing these techniques in different application settings. \commentblock{explain what the main result reveals in direct comparison to what was thought to be the case previously, or how the main result adds to previous knowledge} Our improvements reveal that isogeny based cryptosystems still have much potential for improved performance metrics, particularly in the domain of intelligent implementation. While some practitioners may believe isogeny-based cryptosystems to be inefficient beyond practicality, we show that these systems still have room for improvement, and with continued research can be made more efficient - and eventually practical. \commentblock{provide a broader perspective, readily comprehensible to any technical person} Achieving more efficient implementations for cryptographic protocols will allow us to make them more accessible. With faster and lower-overhead implementations we can implement crypto primitives on low bandwidth, low spec devices; ensuring that more and more devices can be made resistant to attacks from quantum computers.

\end{abstract} 
