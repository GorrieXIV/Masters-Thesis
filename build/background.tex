\chapter{Technical Background}

Over the course of the past decade, Elliptic Curve Cryptography has proven itself a mainstay in the wide world of applied Cryptology. While Isogeny based cryptography does build itself up from the same underlying mathematical concepts as ECC, it simultaneously draws from a deeper, more complicated space of niche  geometric notions and algebraic structures.\\

This chapter will cover the following preliminary topics: isogenies as a mathematical structure, Supersingular Isogeny Diffie Hellman and its related procotols, the Fiat-Shamir construction and its quantum-safe adaptation, and finally isogeny based signature schemes.\\

Our discussion of Isogenies will begin with some basic coverage of the underlying algebra. We will provide the material necessary for the remaining sections as we build up in the level of abstraction; working our way through finite fields, elliptic curves, and finally isogenies and their properties.\\

Once we have presented the necessary algebra, we will divulge the specifics of the supersingular isogeny Diffie-Hellman key-exchange protocol.

\section{Isogenies}

\subsection{<Sub-section title>}

\subsection{<Sub-section title>}
some text\cite{citation-2-name-here}, some more text
\subsection{<Sub-section title>}

\subsection{<Sub-section title>}

Refer figure \ref{fig:label}.

\section{Supersingular Isogeny Diffie-Hellman}

\subsection{Zero-Knowledge Proof of Identity}

\section{Fiat-Shamir Construction}

The Fiat-Shamir Construction (sometimes referred to as the Fiat-Shamir Heuristic,) is used

\subsection{Unruh's Post-Quantum Adaptation}



\section{Isogeny Based Signatures}

\begin{algorithm}
\caption{KeyGen($\lambda$)}\label{euclid}
\begin{algorithmic}[1]
\State Pick a random point S of order $\ell^{e_{A}}_{A}$
\State Compute the isogeny $\phi: E \rightarrow E/\langle S \rangle$
\State pk $\gets (E/\langle S \rangle, \phi(P_{B}), \phi(Q_{B}))$
\State sk $\gets S$
\State \Return (pk,sk)
\end{algorithmic}
\end{algorithm}

\begin{algorithm}
\caption{Sign(sk, $m$)}\label{euclid}
\begin{algorithmic}[1]
\For{\texttt{i = 1..2$\lambda$}}
	\State Pick a random point R of order $\ell^{e_{B}}_{B}$
	\State Compute the isogeny $\psi: E \rightarrow E/\langle R \rangle$
	\State Compute either $\phi' : E/\langle R \rangle \rightarrow E/\langle R,S \rangle$ or $\psi' : E/\langle S \rangle \rightarrow E/\langle R,S \rangle$
	\State $(E_{1},E_{2}) \gets (E/\langle R \rangle, E/\langle R,S \rangle)$
	\State $\texttt{com}_{i} \gets (E_{1}, E_{2})$
	\State $\texttt{ch}_{i,0} \gets_{R} \{0,1\}$
	\State $(\texttt{resp}_{i,0}, \texttt{resp}_{i,1}) \gets ((R,\phi(R)), \psi(S))$
	\If{$\texttt{ch}_{i,0} = 1$}
		\State $\texttt{swap}(\texttt{resp}_{i,0},\texttt{resp}_{i,1})$
	\EndIf
	\State $h_{i,j} \gets G(\texttt{resp}_{i,j})$
\EndFor

\State $J_{1} \parallel ... \parallel J_{2\lambda} \gets H(pk, m, (\texttt{com}_{i})_{i},(\texttt{ch}_{i,j})_{i,j},(h_{i,j})_{i,j})$

\State \Return $\sigma \gets ((\texttt{com}_{i})_{i}, (\texttt{ch}_{i,j})_{i,j}, (h_{i,j})_{i,j}, (\texttt{resp}_{i,J_{i}})_{i})$
\end{algorithmic}
\end{algorithm}

\begin{algorithm}[H]
\caption{Verify(pk, $m$, $\sigma$)}\label{euclid}
\begin{algorithmic}[1]
\State $J_{1} \parallel ... \parallel J_{2\lambda} \gets H(pk, m, (\texttt{com}_{i})_{i},(\texttt{ch}_{i,j})_{i,j},(h_{i,j})_{i,j})$
\For{\texttt{i = 0..2$\lambda$}}
	\State \textbf{check} $h_{i,J_{i}} = G(\texttt{resp}_{i,J_{i}})$
	\If{$\texttt{ch}_{i,J_{i}} = 0$}
		\State Parse $(R,\phi(R)) \gets \texttt{resp}_{i,J_{i}}$
		\State \textbf{check} $(R, \phi(R))$ have order $\ell^{e_{B}}_{B}$
		\State \textbf{check} $R$ generates the kernel of the isogeny $E \rightarrow E_{1}$
		\State \textbf{check} $\phi(R)$ generates the kernel of the isogeny $E/\langle S \rangle \rightarrow E_{2}$
	\Else
		\State Parse $\psi(S) \gets \texttt{resp}_{i,J_{i}}$
		\State \textbf{check} $\psi(S)$ has order $\ell^{e_{A}}_{A}$
		\State \textbf{check} $\psi(S)$ generates the kernel of the isogeny $E_{1} \rightarrow E_{2}$
	\EndIf
\EndFor

\If{all checks succeed}
	\State \Return 1
\EndIf
\end{algorithmic}
\end{algorithm}

If we transcribe the above to the language of the Microsoft SIDH API, we have in essense the following:\\

\begin{algorithm}
\caption{KeyGen($\lambda$)}\label{euclid}
\begin{algorithmic}[1]
\State (pk, sk) $\gets \texttt{KeyGeneration\_B()}$
\State \Return (pk,sk)
\end{algorithmic}
\end{algorithm}

\begin{algorithm}
\caption{Sign(sk, $m$)}\label{euclid}
\begin{algorithmic}[1]
\For{\texttt{i = 1..2$\lambda$}}
	\State $(, R, \psi) \gets \texttt{KeyGeneration\_A(E)}$
	\State $E_{1} \gets E/\langle R \rangle$
	\State $(E_{2},E/\langle R,S \rangle) \gets \texttt{SecretAgreement\_B()}$
	\State $(E_{1},E_{2}) \gets (E/\langle R \rangle, E/\langle R,S \rangle)$
	\State $\texttt{com}[i] \gets (E_{1}, E_{2})$
	\State $\texttt{ch}[i] \gets_{R} \{0,1\}$
	\State $(\texttt{resp}[i]_{0}, \texttt{resp}[i]_{1}) \gets ((R,\phi(R)), \psi(S))$
	
%% this portion was skipped?	
%%	\If{$\texttt{ch}[i] = 1$}
%%		\State $\texttt{swap}(\texttt{resp}[i]_{0},\texttt{resp}[i]_{1})$
%%	\EndIf
%%	\State $h_{i,j} \gets G(\texttt{resp}[i]_{j})$
\EndFor

\State $J_{1} \parallel ... \parallel J_{2\lambda} \gets H(pk, m, (\texttt{com}_{i})_{i},(\texttt{ch}_{i})_{i},(h_{i,j})_{i,j})$

\State \Return $\sigma \gets ((\texttt{com}_{i})_{i}, (\texttt{ch}_{i,j})_{i,j}, (h_{i,j})_{i,j}, ((\texttt{resp})[J_{i}])$
\end{algorithmic}
\end{algorithm}
