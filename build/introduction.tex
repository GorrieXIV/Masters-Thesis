\chapter{Introduction}

The past 30 years have brought with them astonishing developments in the field of Quantum Computing. As physicists and engineers race towards error-free and energy efficient implementations of quantum computers, we steadfastly approach a New Age for the art and science of Cryptography.  have come also remarkably. Quantum computers 

\section{Motivation}

Our aim is to improve the efficiency (thus improving the practicality) of isogeny based schemes. More specifically, we will be investigating the C implementation of the Yoo et al. isogeny based signature scheme. 

\subsection{Related Works}

Post-quantum cryptography has been the subject of rigorous and bountiful research for over a decade now. Progress in the subfield of Isogeny-based cryptography has been made slowly yet surely in this time,  

\section{Contributions}

Our explicit contributions to the implementation of this protocol are twofold. Our first contribution involves the implementation of a procedure for batching . This scheme leverages the already parallel nature of the protocol, and.

Additionally, because the compression algorithm in question is itself tenable two contributions can be 


\section{Structure}

This dissertation is divided into 5 main sections. This section and the one that follows contain the relevent preliminary information for understanding the contributions of the thesis. The two following sections thereafter outline in detail the two contributions of this dissertation. The 5$^{th}$ and final section concludes the dissertation while offering final remarks and suggestions for continued research. 

\subsection{Layout}

Over the course of the past decade, elliptic curve cryptography (ECC) has proven itself a mainstay in the wide world of applied cryptology. While isogeny based cryptography does build itself up from the same underlying field of mathematics as ECC, it simultaneously draws from a slightly more complicated space of algebraic notions. Much of this chapter will be dedicated to illuminating these notions in a manner that should be digestable for those without serious background in algebraic geometry, or abstract algebra in general.

\subsection{Notation}

To denote high-level procedures, we write their titls in \textbf{bold}. Procedures intended for machine execution are titled with \code{monospace} font - and we use the same notation for C modules, functions, and variable name. 